\documentclass[12pt]{article}
\usepackage{amsmath,amssymb,amsthm}
\renewcommand{\theequation}{\alph{equation}}
\usepackage[pdftex]{graphicx}
\usepackage[shortlabels]{enumitem}
\usepackage[inline]{asymptote}
\usepackage{fancyhdr}
\usepackage{mathtools}
\usepackage[english]{babel}
\usepackage[margin=1 in]{geometry}
\usepackage[nottoc]{tocbibind}
\usepackage{hyperref}
\usepackage{setspace}
\usepackage{csquotes}
\usepackage[final]{pdfpages}
\usepackage[version=4]{mhchem}
\usepackage{siunitx}
\usepackage[T1]{fontenc}
\usepackage{empheq}
\usepackage{mdframed}
\newcommand{\standard}{^\circ}
\pagestyle{fancy}
\lhead{Kinetics Notes}
\rhead{\thepage}
\newcommand{\incfig}[1]{%
    \def\svgwidth{\columnwidth}
    \import{images/}{#1.pdf_tex}
}
\begin{document}

\section{Basic Terms}  
Before we can begin giving notes, we need to learn some key terms that are necessary to know before solving problems in chemical kinetics.
\begin{itemize}
    \item \textbf{Kinetics:} The rate at which chemical reactions happens. Kinetics deals with studying how fast a reaction occurs, and how much energy it takes to occur.
    \item \textbf{Order:} The amount of effect a substance has on the rate of a reaction
    \item \textbf{Rate Law:} The mathematical relationship of the concentration of reactants against the concentration of products
    \item \textbf{}
\end{itemize}
    

\end{document}