\documentclass[11pt]{article}
\usepackage{amsmath,amssymb,amsthm}
% \renewcommand{\theequation}{\alph{equation}}
\usepackage[pdftex]{graphicx}
\usepackage[shortlabels]{enumitem}
\usepackage[inline]{asymptote}
\usepackage{fancyhdr}
\usepackage{mathtools}
\usepackage[english]{babel}
\usepackage[margin=1 in]{geometry}
\usepackage[nottoc]{tocbibind}
\usepackage{hyperref}
\usepackage{setspace}
\usepackage{csquotes}
\usepackage[final]{pdfpages}
\usepackage[version=4]{mhchem}
\usepackage{siunitx}
\usepackage[T1]{fontenc}
\usepackage{empheq}
\usepackage{mdframed}
\newcommand{\standard}{^\circ}
\pagestyle{fancy}
\lhead{Kinetics Notes}
\rhead{\thepage}
\newcommand{\incfig}[1]{%
    \def\svgwidth{\columnwidth}
    \import{images/}{#1.pdf_tex}
}
\begin{document}

\section{Basic Terms}  Before we can begin giving notes, we need to learn some key terms that are necessary to know before solving problems in chemical kinetics.
\begin{itemize}
    \item \textbf{Kinetics:} The rate at which chemical reactions happens. Kinetics deals with studying how fast a reaction occurs, and how much energy it takes to occur.
    \item \textbf{Order:} The amount of effect a substance has on the rate of a reaction.
    \item \textbf{Rate Law:} The mathematical relationship of the concentration of reactants against the concentration of products.
    \item \textbf{Activation Energy:} The energy necessary for a reaction to occur. For example, molecules will need to collide at a certain 
    speed for them to react, which requires a certain amount of internal energy.
    \item \textbf{Rate Constant:} A factor found in rate laws, represented by $k$
    \item \textbf{Half-life} The time for a substance to decrease to 50\% of its original concentration. Represented by $t_{1/2}$.
\end{itemize}
\section{The General Rate Law} For a reaction 

$$\ce{aA + bB -> cC + dD}$$
The general rate law says 
\begin{align}
    -\frac{1}{\text{a}}\left(\frac{d[\text{A}]}{dt}\right) = -\frac{1}{\text{b}}\left(\frac{d[\text{B}]}{dt}\right) = \frac{1}{\text{c}}\left(\frac{d[\text{C}]}{dt}\right) = \frac{1}{\text{d}}\left(\frac{d[\text{D}]}{dt}\right)
\end{align}
Where the lowercase letters represent the stoichiometric coefficients of each substance. Note the negative before the rates for substance A and B. This is because they are consumed, while C and D are produced.



    

\end{document}